% !TEX encoding = UTF-8 Unicode
% -*- coding: UTF-8; -*-
\ifdefined\ishandout
\documentclass[handout]{beamer}
\else
\documentclass[11pt]{beamer}
\fi

\usepackage[frenchb]{babel}
\usepackage[T1]{fontenc}
\usepackage[utf8]{inputenc}
\usepackage{hyperref}
\usepackage{multirow}
\usepackage{listings}
\usepackage{fancyvrb}
\usepackage{tikz}
\usepackage{framed}
\usepackage{algorithm}
\usepackage{algorithmic}
\usepackage{xcolor}
\usepackage{color, colortbl}
\ifdefined\ishandout
\usepackage{handoutWithNotes}
\fi
\usepackage{slashbox}
\usepackage{amsmath}
\usepackage{bm}
\usepackage{hhline}
\usepackage{xmpmulti}

\usetikzlibrary{shapes.geometric}
\usetikzlibrary{positioning}
\usetikzlibrary{shapes.arrows, chains}
\usetikzlibrary{arrows,calc}
\usetikzlibrary{shapes.multipart}
\usepackage{array}
\usetheme{Boadilla}

\usefonttheme[onlymath]{serif}

\newcommand{\R}{\mathbb{R}}
\newcommand{\C}{\mathbb{C}}
\newcommand{\N}{\mathbb{N}}
\newcommand{\Z}{\mathbb{Z}}
\newcommand{\E}{\mathbb{E}}
\newcommand{\Var}{\text{Var}}
\newcommand{\Cov}{\text{Cov}}
\ifdefined\ishandout
\pgfpagesuselayout{3 on 1 with notes}[a4paper,border shrink=5mm]
\usecolortheme{dove}
\else
%\usecolortheme{dolphin}
\usecolortheme{beaver}
\fi


\lstnewenvironment{codeC}
{ \lstset{language=C,
    otherkeywords={printf,scanf}}
}
{}

\ifdefined\ishandout
\definecolor{mygreen}{rgb}{0,0,0}
\definecolor{mymauve}{rgb}{0,0,0}
\definecolor{myblue}{rgb}{0,0,0}
\else
\definecolor{mygreen}{rgb}{0,0.6,0}
\definecolor{mymauve}{rgb}{0.58,0,0.82}
\definecolor{myblue}{rgb}{0,0,1}

\fi

%% Notes
%\setbeameroption{show only notes}


\definecolor{mygray}{rgb}{0.5,0.5,0.5}

\lstset{ language=Python,%
  backgroundcolor=\color{white},   % choose the background color; you must add \usepackage{color} or \usepackage{xcolor}
  basicstyle=\footnotesize,        % the size of the fonts that are used for the code
  breakatwhitespace=false,         % sets if automatic breaks should only happen at whitespace
  breaklines=true,                 % sets automatic line breaking
  captionpos=b,                    % sets the caption-position to bottom
  commentstyle=\color{mygreen},    % comment style
  deletekeywords={...},            % if you want to delete keywords from the given language
  escapeinside={\%*}{*)},          % if you want to add LaTeX within your code
  extendedchars=true,              % lets you use non-ASCII characters; for 8-bits encodings only, does not work with UTF-8
  frame=tb,	                   % adds a frame around the code
  keepspaces=true,                 % keeps spaces in text, useful for keeping indentation of code (possibly needs columns=flexible)
  keywordstyle=\color{blue},       % keyword style
  otherkeywords={*,...},           % if you want to add more keywords to the set
  numbers=none,                    % where to put the line-numbers; possible values are (none, left, right)
  numbersep=5pt,                   % how far the line-numbers are from the code
  numberstyle=\tiny\color{mygray}, % the style that is used for the line-numbers
  rulecolor=\color{black},         % if not set, the frame-color may be changed on line-breaks within not-black text (e.g. comments (green here))
  showspaces=false,                % show spaces everywhere adding particular underscores; it overrides 'showstringspaces'
  showstringspaces=false,          % underline spaces within strings only
  showtabs=false,                  % show tabs within strings adding particular underscores
  stepnumber=2,                    % the step between two line-numbers. If it's 1, each line will be numbered
  stringstyle=\color{mymauve},     % string literal style
  tabsize=3,	                   % sets default tabsize to 2 spaces
  title=\lstname                   % show the filename of files included with \lstinputlisting; also try caption instead of title
}
%\lstset{language=Python,
% breakatwhitespace=false,         % sets if automatic breaks should only happen at whitespace
%  breaklines=true,                 % sets automatic line breaking
%  captionpos=b,                
%%commentstyle=\itshape\color{mymauve},
%%keywordstyle=\bfseries\color{myblue},
%numbers=left,                    % where to put the line-numbers; possible values are (none, left, right)
%  numbersep=8pt,                   % how far the line-numbers are from the code
%  numberstyle=\tiny\color{mygray}, % the style that is used for the line-numbers
%%  rulecolor=\color{black},         % if not set, the frame-color may be changed on line-breaks within not-black text (e.g. comments (green here))
%  showspaces=false,                % show spaces everywhere adding particular underscores; it overrides 'showstringspaces'
%%  showstringspaces=false,          % underline spaces within strings only
%  showtabs=false,                  % show tabs within strings adding particular underscores
%  stepnumber=2,                    % the step between two line-numbers. If it's 1, each line will be numbered
%%  stringstyle=\color{mygreen},     % string literal style
%  tabsize=2 
%}
\ifdefined\ishandout
\newcommand{\red}{\textbf}
\else
\newcommand{\red}{\textcolor{red}}
\fi
%\newcommand \emph
%Default size : 12.8 cm * 9.6 cm

\newcommand{\tmark}[1]{\tikz[remember picture, baseline=-.5ex]{\coordinate(#1);}}

\ifdefined\ishandout
\newenvironment<>{codeblock}[1]{%begin
  \setbeamercolor{block title}{fg=black,bg=lightgray!80}%
  \begin{block}{#1}}
  % \begin{codeC}}
  %  {\end{codeC}
{  
\end{block}}

\newenvironment<>{termblock}[1]{
    \setbeamercolor{block title}{fg=black,bg=lightgray!90}%
    \begin{block}{#1}
}
%     \begin{Verbatim}}
{%\end{Verbatim}
\end{block}
}

\definecolor{bluegreen}{RGB}{0,0,0}
%\definecolor{bluegreen}{rgb}{0,0.6,0.8}
\else

\newenvironment<>{codeblock}[1]{%begin
  \setbeamercolor{block title}{fg=darkgray,bg=yellow}%
  \begin{block}{#1}}
  % \begin{codeC}}
  %  {\end{codeC}
{  
\end{block}}

\newenvironment<>{termblock}[1]{
    \setbeamercolor{block title}{fg=white,bg=lightgray}%
    \begin{block}{#1}}
%     \begin{Verbatim}}
{%\end{Verbatim}
\end{block}
}

\definecolor{bluegreen}{RGB}{0,149,182}
%\definecolor{bluegreen}{rgb}{0,0.6,0.8}
\fi

%\newcommand{\output}[1]{
\setbeamertemplate{navigation symbols}{}
\newcommand{\bvrb}{\Verb[commandchars=£µ§,formatcom=\color{bluegreen}]}
\newcommand{\footvrb}{\footnotesize\Verb}
\newcommand{\vrbalert}[2][]{\visible<#1>{#2}}
%%% Commande pour les listes/arbres
\newcommand{\mvide}{\nodepart{one} \nodepart{two}}
\newcommand{\tvide}{\nodepart{one} \nodepart{two} \nodepart{three}}
\newcommand{\rref}[1][]{\hfill{\scriptsize\textit{#1}}}


\newcommand{\odif}[2]{\frac{d #1}{d #2}} 
%%Fin des commandes pour les listes/arbres.
\newcommand{\gooditem}[1]{\setbeamercolor{item}{fg=green}\item #1} 
\newcommand{\pooritem}[1]{\setbeamercolor{item}{fg=red}\item #1} 
\setbeamerfont{caption}{size=\scriptsize}

%%% Paramètres du cours (à régler)
%Numéro du cours
\newcommand{\nb}{1}

\title[lagrangian assimilation]{Assimilation of non-conventional observations. Application to the estimating of ocean surface currents.}
\author[J. Brajard]{julien.brajard@upmc.fr}
\institute[LOCEAN/UPMC]{LOCEAN-UPMC}
\date{29 May 2017}
\begin{document}
\tikzstyle{every picture}+=[remember picture]

%%%%%%%%%%%%%%%%%%%%%%
\begin{frame}
\frametitle{Position of the problem}
\begin{columns}
\column{.5\textwidth}
  \begin{figure}
 \begin{center}
 \includegraphics[width=\textwidth]{./fig/chart1.png}
\end{center}
 \end{figure}
 \column{.5\textwidth}
 \begin{figure}
 \begin{center}
\includegraphics[width=\textwidth]{./fig/Euler_LagrangeNew.png}
\end{center}
 \end{figure}
 \end{columns}
 
\begin{itemize}
\item $\mathcal{M}$ is a parametrization of non-geostrophic effects (wind)
\item $\mathcal{G}$ is the advection operator solving
  $\odif{\vec{r}}{t} =\vec{U}(\vec{r}(t),t)$
\end{itemize}
\begin{alertblock}{Objective}
From observed $\vec{r}_i$, estimate the velocity field $\vec{U}_0$
\end{alertblock}
 \rref[Issa et al. (2016)]

\end{frame}


\begin{frame}{Results using real data}
Using the velocity field obtained by assimilating 2 drifters to
forecast the trajectory of a 3rd drifter (not used in the assimilation).\\
\vspace{3em}
\centering
\includegraphics[width=0.7\textwidth]{fig/ReconstructedCNRSExp_bmtogreen_2days_average_zoom.png}
\end{frame}


%%%%%%%%%%%%%%%%%%%%%%%%%
\begin{frame}
\frametitle{Summary of the method}
\begin{columns}[b]
\column{.5\textwidth}
\begin{figure}
\begin{tikzpicture}[
auto,
node distance = 0.4cm,
>=stealth',
weight/.style={rectangle,minimum height = .65cm, draw,fill=orange!20},
param/.style={rectangle,minimum height = .65cm, draw,fill=cyan!20},
neuron/.style={circle,minimum height = .65cm,draw,fill=magenta!20}]
\node [neuron] (plus) {$+$} ;
\node [neuron] (m) [above=of plus]{$\mathcal{M}$} edge [<-] (plus);
\node [neuron] (W) [left= of plus]{$\mathcal{W}$} edge [->] (m);
\node [param] (u) [below=of plus] {$u_b(t)$} edge [->] (plus);
\node[param] (w) [left=of u]{$w(t)$} edge[->] (W);
\node[weight] (a) [left=of w] {$a_w$} edge[->](W);
\node [param] (r) [left=of a] {$r_i(t)$} ;
\path [->,draw] (r) |- (m) ;
\node [weight] (du) [right=of u]{$\delta u$} edge [->] (plus);
\node [param] (y) [above=of m]{$r_i(t+1)$} edge [<-] (m);
\end{tikzpicture}
\caption{Physical network}
\end{figure}
\column{.5\textwidth}
\begin{onlyenv}<2>
\begin{block}{Summary}
\begin{itemize}
\item Very simple numerical model
\gooditem Makes use of the lagrangian information
\gooditem Improve eulerian velocity field
\pooritem No predictive power
\pooritem Uncertainty has to be evaluated
\pooritem Many "hyper-parameters"
\end{itemize}
\end{block}
\end{onlyenv}
\begin{onlyenv}<3>
\begin{figure}
\begin{tikzpicture}[
auto,
node distance = 0.4cm,
>=stealth',
weight/.style={rectangle,minimum height = .65cm, draw,fill=orange!20},
param/.style={rectangle,minimum height = .65cm, draw,fill=cyan!20},
neuron/.style={circle,minimum height = .65cm,draw,fill=magenta!20}]
\node [neuron] (plus) {} ;
\node [neuron] (m) [above=of plus]{} edge [<-] (plus);
\node [neuron] (R) [right= of plus] {} edge [->] (m);
\node [neuron] (W) [left= of plus]{} edge [->] (m);
\node[param] (w) [below=of plus]{$w(t)$} edge[->] (plus);

\node [param] (u) [right=of w] {$u_b(t)$} edge [->] (R);
%\node[weight] (a) [left=of w] {$a_w$} edge[->](W);
\node [param] (r) [left=of w] {$r_i(t)$} edge[->](W) ;
\node[weight] (we) [right= of u] {$\bf{W}$};
\path [->,draw] (we) -- (R) ;
\path [->,draw] (we) -- (plus) ;
\path [->,draw] (we) -- (W) ;
\path [->,draw] (we) edge [bend right] (m);

%\node [weight] (du) [right=of u]{$\delta u$} edge [->] (plus);
\node [param] (y) [above=of m]{$r_i(t+1)$} edge [<-] (m);
\end{tikzpicture}
\caption{Artificial neural networks (machine learning)}
\end{figure}
\end{onlyenv}
\end{columns}
\end{frame}



\end{document}