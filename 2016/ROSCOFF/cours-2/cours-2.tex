\ifdefined\ishandout
\documentclass[handout]{beamer}
\else
\documentclass{beamer}
\fi

\usepackage[frenchb]{babel}
\usepackage[T1]{fontenc}
\usepackage[latin1]{inputenc}
\usepackage{hyperref}
\usepackage{multirow}
\usepackage{listings}
\usepackage{fancyvrb}
\usepackage{tikz}
\usepackage{framed}
\usepackage{algorithm}
\usepackage{algorithmic}
\usepackage{xcolor}
\usepackage{color, colortbl}
\ifdefined\ishandout
\usepackage{handoutWithNotes}
\fi

\usepackage{amsmath}
\usepackage{bm}

\usetikzlibrary{shapes.geometric}
\usetikzlibrary{positioning}
\usetikzlibrary{shapes.arrows, chains}
\usetikzlibrary{arrows,calc}
\usetikzlibrary{shapes.multipart}
\usepackage{array}
\usetheme{Boadilla}

\usefonttheme[onlymath]{serif}

\newcommand{\R}{\mathbb{R}}
\newcommand{\C}{\mathbb{C}}
\newcommand{\N}{\mathbb{N}}
\newcommand{\Z}{\mathbb{Z}}


\ifdefined\ishandout
\pgfpagesuselayout{3 on 1 with notes}[a4paper,border shrink=5mm]
\usecolortheme{dove}
\else
%\usecolortheme{dolphin}
\usecolortheme{beaver}
\fi


\lstnewenvironment{codeC}
{ \lstset{language=C,
    otherkeywords={printf,scanf}}
}
{}

\ifdefined\ishandout
\definecolor{mygreen}{rgb}{0,0,0}
\definecolor{mymauve}{rgb}{0,0,0}
\definecolor{myblue}{rgb}{0,0,0}
\else
\definecolor{mygreen}{rgb}{0,0.6,0}
\definecolor{mymauve}{rgb}{0.58,0,0.82}
\definecolor{myblue}{rgb}{0,0,1}

\fi

\definecolor{mygray}{rgb}{0.5,0.5,0.5}

\lstset{ language=Python,%
  backgroundcolor=\color{white},   % choose the background color; you must add \usepackage{color} or \usepackage{xcolor}
  basicstyle=\footnotesize,        % the size of the fonts that are used for the code
  breakatwhitespace=false,         % sets if automatic breaks should only happen at whitespace
  breaklines=true,                 % sets automatic line breaking
  captionpos=b,                    % sets the caption-position to bottom
  commentstyle=\color{mygreen},    % comment style
  deletekeywords={...},            % if you want to delete keywords from the given language
  escapeinside={\%*}{*)},          % if you want to add LaTeX within your code
  extendedchars=true,              % lets you use non-ASCII characters; for 8-bits encodings only, does not work with UTF-8
  frame=tb,	                   % adds a frame around the code
  keepspaces=true,                 % keeps spaces in text, useful for keeping indentation of code (possibly needs columns=flexible)
  keywordstyle=\color{blue},       % keyword style
  otherkeywords={*,...},           % if you want to add more keywords to the set
  numbers=none,                    % where to put the line-numbers; possible values are (none, left, right)
  numbersep=5pt,                   % how far the line-numbers are from the code
  numberstyle=\tiny\color{mygray}, % the style that is used for the line-numbers
  rulecolor=\color{black},         % if not set, the frame-color may be changed on line-breaks within not-black text (e.g. comments (green here))
  showspaces=false,                % show spaces everywhere adding particular underscores; it overrides 'showstringspaces'
  showstringspaces=false,          % underline spaces within strings only
  showtabs=false,                  % show tabs within strings adding particular underscores
  stepnumber=2,                    % the step between two line-numbers. If it's 1, each line will be numbered
  stringstyle=\color{mymauve},     % string literal style
  tabsize=3,	                   % sets default tabsize to 2 spaces
  title=\lstname                   % show the filename of files included with \lstinputlisting; also try caption instead of title
}
%\lstset{language=Python,
% breakatwhitespace=false,         % sets if automatic breaks should only happen at whitespace
%  breaklines=true,                 % sets automatic line breaking
%  captionpos=b,                
%%commentstyle=\itshape\color{mymauve},
%%keywordstyle=\bfseries\color{myblue},
%numbers=left,                    % where to put the line-numbers; possible values are (none, left, right)
%  numbersep=8pt,                   % how far the line-numbers are from the code
%  numberstyle=\tiny\color{mygray}, % the style that is used for the line-numbers
%%  rulecolor=\color{black},         % if not set, the frame-color may be changed on line-breaks within not-black text (e.g. comments (green here))
%  showspaces=false,                % show spaces everywhere adding particular underscores; it overrides 'showstringspaces'
%%  showstringspaces=false,          % underline spaces within strings only
%  showtabs=false,                  % show tabs within strings adding particular underscores
%  stepnumber=2,                    % the step between two line-numbers. If it's 1, each line will be numbered
%%  stringstyle=\color{mygreen},     % string literal style
%  tabsize=2 
%}
\ifdefined\ishandout
\newcommand{\red}{\textbf}
\else
\newcommand{\red}{\textcolor{red}}
\fi
%\newcommand \emph
%Default size : 12.8 cm * 9.6 cm

\newcommand{\tmark}[1]{\tikz[remember picture, baseline=-.5ex]{\coordinate(#1);}}

\ifdefined\ishandout
\newenvironment<>{codeblock}[1]{%begin
  \setbeamercolor{block title}{fg=black,bg=lightgray!80}%
  \begin{block}{#1}}
  % \begin{codeC}}
  %  {\end{codeC}
{  
\end{block}}

\newenvironment<>{termblock}[1]{
    \setbeamercolor{block title}{fg=black,bg=lightgray!90}%
    \begin{block}{#1}
}
%     \begin{Verbatim}}
{%\end{Verbatim}
\end{block}
}

\definecolor{bluegreen}{RGB}{0,0,0}
%\definecolor{bluegreen}{rgb}{0,0.6,0.8}
\else

\newenvironment<>{codeblock}[1]{%begin
  \setbeamercolor{block title}{fg=darkgray,bg=yellow}%
  \begin{block}{#1}}
  % \begin{codeC}}
  %  {\end{codeC}
{  
\end{block}}

\newenvironment<>{termblock}[1]{
    \setbeamercolor{block title}{fg=white,bg=lightgray}%
    \begin{block}{#1}}
%     \begin{Verbatim}}
{%\end{Verbatim}
\end{block}
}

\definecolor{bluegreen}{RGB}{0,149,182}
%\definecolor{bluegreen}{rgb}{0,0.6,0.8}
\fi

%\newcommand{\output}[1]{
\setbeamertemplate{navigation symbols}{}
\newcommand{\bvrb}{\Verb[commandchars=£µ§,formatcom=\color{bluegreen}]}
\newcommand{\footvrb}{\footnotesize\Verb}
\newcommand{\vrbalert}[2][]{\visible<#1>{#2}}
%%% Commande pour les listes/arbres
\newcommand{\mvide}{\nodepart{one} \nodepart{two}}
\newcommand{\tvide}{\nodepart{one} \nodepart{two} \nodepart{three}}

%%Fin des commandes pour les listes/arbres.



%%% Paramètres du cours (à régler)
%Numéro du cours
\newcommand{\nb}{1}

\title[Linear Algebra]{Basics of Linear Algebra}
\author[]{julien.brajard@upmc.fr}
\institute[UPMC]{UPMC}
\date{1-5 August 2016}
\begin{document}
%%%%%%%%%%%%%%%%%%%%% SLIDES DE TITRE
\begin{frame}
\titlepage
%\centering{
%\url{http://australe.upmc.fr} (onglet EPU-C5-IGE Info Gen)}
\end{frame}
%%%%%%%%%%%%%%%%%%%%%

%%%%%%%%%%%%%%%%%%%%%%%%%%%%%%%%%%%%%%%%%%%%%%%%%%%%%%%%%%%%%%%%%%%%%%%%%%%%%%%%%%%%%%

\begin{frame}
\frametitle{Scalars, Vectors, Matrices (and Tensors)}
We define some mathematical objects :
\begin{itemize}
\item \alert{Scalars}: a single number usually real ($\in \R$)
\item \alert{Vectors}: An array of numbers. For example, $\bm{x} \in \R^n$
is a vector containing $n$ elements, the i-th element
is denoted $x_i$.
\item \alert{Matrices}: A 2-D array of numbers. For exemple,
$\bm{A} \in \R^{m\times n}$ is a matrix containing $m$ rows and 
$n$ columns. The element in row $i$ and column $j$ is denoted $A_{i,j}$.
\item \alert{Tensors}: A generalization of matrix with more than two dimensions.
In 3-D, an element of the tensor $\bm{A}$ is denoted $A_{i,j,k}$.
\end{itemize}

\end{frame}
%%%%%%%%%%%%%%%%%%%%%%%%%%%%%%%%%%%%%%%%%%%%%%%%%%%%%%%%%%%%%%%%%%%%%%%%%%%%%%%%%%%%%%

\begin{frame}
\frametitle{Operations on matrix and vectors}
\begin{itemize}
\item The Transpose $^T$ operation is defined by : $(A^T)_{i,j}=A_{j,i}$
\item Sum of matrix (or vectors) : $\bm{C} = \bm{A} + \bm{B}$ is defined by:\\
$C_{i,j} = A_{i,j}+B_{i,j}$
\end{itemize}
\end{frame}

%%%%%%%%%%%%%%%%%%%%%%%%%%%%%%%%%%%%%%%%%%%%%%%%%%%%%%%%%%%%%%%%%%%%%%%%%%%%%%%%%%%%%%

\begin{frame}
\frametitle{Identity and Inverse matrix}

\end{frame}


%%%%%%%%%%%%%%%%%%%%%%%%%%%%%%%%%%%%%%%%%%%%%%%%%%%%%%%%%%%%%%%%%%%%%%%%%%%%%%%%%%%%%%

\begin{frame}
\frametitle{Norms}

\end{frame}


\end{document} 
%%%%%%%%%%%%%%%%%%%%% SECTION 1
\section{Les algorithmes}\label{section:1}
\begin{frame}
\begin{columns}
        \column{4.8cm}
            \tableofcontents[currentsection]
        \column{7cm}
        \centering{
            \includegraphics[width=7cm]{fig/Algorithm-sheldon.png}
            }
                 \textit{ I believe I've isolateblblblblblblsblbslbslbsl
            sblbslblsblsblblsblbs
            lbslblbslsb d the algorithm for making friends.}
     
            
            \small{
            \hfill Sheldon Cooper, 
            
            \hfill in \textit{The Big Band Theory}, Season 2, Episode 13
            }


    \end{columns}

\end{frame}


%%%%%%%%%%%%%%%%%%%%%
\subsection{Introduction}
    \begin{frame}
    \frametitle{Pourquoi faire appel à des algorithmes ?}
    Pour automatiser des tâches
    
    Exemples :
    \begin{itemize}
    \item Métier à tisser\\
    \item Méthode de calcul à la main d'une division\\
    \item Recette de cuisine\\
    \item ...\\
    \end{itemize}
    \end{frame}
 
 %%%%%%%%%%%%%%%%%
 
    \begin{frame}
    \frametitle{Qu'est-ce qu'un algorithme ?}
    \begin{block}{Définition}
    Un algorithme est un ensemble 
    ordonné d'instructions simples
permettant de résoudre un problème.
    \end{block}
    \end{frame}
    
 %%%%%%%%%%%%%%%%%%
 \subsection{Construction d'un algorithme}
%%%%%%%%%%%%%%%%%%%    
\section{La machine de Turing}
%%%%%%%%%%%%%%%%%%%%
 
  
\begin{frame}[fragile]
\frametitle{Un peu d'histoire...}
\begin{codeblock}{Test}
\begin{codeC}
for (int i = 0 ; i < n ; i ++) {
    //a comment
    printf("%d",i);
    }
\end{codeC}
\end{codeblock}

\begin{termblock}{test 2}
\lstset{escapeinside={§§}}
\begin{lstlisting}
§\textbf{>>}§./a.out
§\color{darkgray}{\texttt{  Hello World}}§
\end{lstlisting}
\end{termblock}

 \begin{block}{Bloc standard}
blablabla
\end{block}
\end{frame}


\begin{frame}[fragile]
\frametitle{essai}
\begin{columns}
\column{6cm}
\begin{block}

\begin{figure}
\begin{tikzpicture} [
    auto,
    decision/.style = { diamond, draw=blue, thick, fill=blue!20,
                        text width=5em, text badly centered,
                        inner sep=1pt, rounded corners },
    block/.style    = { rectangle, draw=blue, thick, 
                        fill=blue!20, text width=10em, text centered,
                        rounded corners, minimum height=2em },
    line/.style     = { draw, thick, ->, shorten >=2pt },
  ]
   \matrix [column sep=-10mm, row sep=10mm] {
                    & \node [text centered] (x) {$\mathbf{X}$};            & \\
                    & \node (null1) {};                                    & \\
                    & \node [block] (doa) {\textsf{DoAE}($\mathbf{X}$)};   & \\
  	               \node(null3){}; & \node [decision] (uiddes)
                        {\textsf{UID}($\hat{\mathbf{X}}$)};
                                  & \node[text centered](tra){$\mathbf{i}$}; \\
                  & \node [block] (track) {\textsf{DoAT}($\mathbf{x}$)}; & \\
                    & \node [block] (pesos)
                        {\textsf{BF}(DoA$_{\mathrm{T}}$,DoAs)};            & \\
                    & \node [block] (filtrado)
                        {\textsf{SF}($\mathbf{w}$,$\mathbf{x}$)};          & \\
                    & \node [text centered] (xf) {$\hat{x}(t)$ };          & \\
  };
  % connect all nodes defined above
 \begin{scope} [every path/.style=line]
    \path (x)        --    (doa);
    \path (doa)      --    node [near start] {DoAs} (uiddes);
    \path (tra)      --    (uiddes);
    \path (uiddes)   --++  (-3,0) node [near start] {no} |- (null1);
    \path (uiddes)   --    node [near start] {DoA} (track);
    \path (track)    --    node [near start] {DoA$_{\mathrm{T}}$} (pesos);
    \path (pesos)    --    node [near start] {\textbf{w}} (filtrado);
    \path (filtrado) --    (xf);
  
  \end{scope}
\end{tikzpicture}
\end{figure}
\end{block}
\column{3cm}
\begin{block}{bulbul}
\end{block}
\end{columns}
\end{frame}

\end{document}
