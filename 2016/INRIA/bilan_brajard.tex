\documentclass[10pt]{beamer}

\usepackage[frenchb]{babel}
\usepackage[T1]{fontenc}
\usepackage[utf8]{inputenc}
\usepackage{hyperref}
\usepackage{listings}
\usepackage{fancyvrb}
\usepackage{tikz}
\usepackage{framed}
\usepackage{algorithm}
\usepackage{algorithmic}
  \usepackage{amsmath,amssymb,amsthm}
  \usepackage{dsfont}\let\mathbb\mathds
\usepackage{setspace}

\usetikzlibrary{shapes.geometric}
\usetikzlibrary{shapes.arrows}
\usetikzlibrary{arrows}
\usepackage{array}

%\usetheme{Boadilla}
\usetheme{inria}
\usepackage{helvet}
\usecolortheme{dolphin}
\newcommand{\R}{\mathbb{R}}
\newcommand{\C}{\mathbb{C}}
\newcommand{\N}{\mathbb{N}}
\newcommand{\Z}{\mathbb{Z}}


\newcommand{\inriaswitchcolors}[1]{%
\pgfaliasimage{figfootline}{figfootline-#1}% !!!
\pgfaliasimage{figbackground}{figbackground-#1}% !!!
\pgfaliasimage{figbackground}{figbackground-#1}% !!!
}

%\inriaswitchcolors{pastelgreen}
\lstnewenvironment{codeC}
{ \lstset{language=C,
    otherkeywords={printf,scan}}
}
{}
\newcommand{\red}{\textcolor{red}}
\newcommand{\plus}[1]{\textcolor{orange}{\textbf{#1}}}
%\newcommand \emph
%Default size : 12.8 cm * 9.6 cm

\newenvironment<>{codeblock}[1]{%begin
  \setbeamercolor{block title}{fg=darkgray,bg=yellow}%
  \begin{block}{#1}}
  % \begin{codeC}}
  %  {\end{codeC}
{  
\end{block}}

\newenvironment<>{termblock}[1]{
    \setbeamercolor{block title}{fg=white,bg=lightgray}%
    \begin{block}{#1}}
%     \begin{Verbatim}}
{%\end{Verbatim}
\end{block}
}
%\newcommand{\output}[1]{

%%% Paramètres du cours (à régler)
%Numéro du cours
\newcommand{\nb}{1}
\setbeamertemplate{navigation symbols}{}%remove navigation symbols

\title[Bilan/perspective]{Délégation à CLIME : \mbox{Bilan/perspective}} %mbox insecable
\subtitle{2014-2016}
\author[J. Brajard]{Julien Brajard}
\institute[CLIME]{Inria CLIME}
\date{18 Septembre 2015}
\begin{document}
%%%%%%%%%%%%%%%%%%%%% SLIDES DE TITRE
\begin{frame}
\titlepage
\end{frame}
%%%%%%%%%%%%%%%%%%%%%
\begin{frame}
\frametitle{Contexte de cette présentation}

\begin {block}{}
La méthode de renormalisation est une méthode proposé dans~\cite{issartel2007} pour la détection de source dans la dispersion atmosphérique.
\end{block}
\bibliographystyle{apalike} 
\bibliography{bib-assim.bib}
My pgf version is: \pgfversion
\end{frame}

\begin{frame}
\begin{figure}
\begin{tikzpicture}
  \begin{scope}[blend group = soft light]
    \fill[red!40!white]   ( 90:1) circle (2.5);
    \fill[green!40!white] (210:1) circle (2.5);
    \fill[blue!40!white]  (330:1) circle (2.5);
  \end{scope}
  \node at ( 90:2.7)    {Inria};
  \node at ( 210:2.7)   {LIP6};
  \node at ( 330:2.7)   {LOCEAN};
  \node [text centered, text width=80pt,font=\small] {\textbf{Probl\'ematique g\'eophysique}\\ (ex : Mer M\'editerran\'ee)};
  \node at (30:2.2) [text centered,text width = 40pt,font=\scriptsize, anchor=center] {Stage Inria-LOCEAN};
  \node at (150:2.2) [text centered,text width = 40pt,font=\scriptsize\scriptsize, anchor=center] {Stage Inria-LIP6};
  \node at (270:2.2) [text centered,text width = 40pt,font=\scriptsize, anchor=center] {Stage LIP6-LOCEAN};

\end{tikzpicture}
\end{figure}
\end{frame}
\end{document}
